%        File: TDDA69-lectures.tex
%     Created: Tue Jan 26 01:00 PM 2016 C
% Last Change: Tue Jan 26 01:00 PM 2016 C
%
\documentclass[a4paper]{article}
\usepackage[utf8]{inputenc}
\usepackage{listings}
\newtheorem{definition}{Definition}
\newtheorem{lemma}{Lemma}
\title{TDDA69: Program and Data Structures\\Lecture Notes}
\author{Pontus Persson}
\date{VT-2016}
\begin{document}
\maketitle
\tableofcontents
\section{Imperative programming and data structures}
\marginpar{\emph{Lecture 2\\2016-01-26}}
\subsection{Imperative programming}
Programming where you express how computations are executed (cooking recipe).
Examples: C/C++, Pascal\\
Most hardware follow the Con Neumann architecture, which is imperative.
Many early languages were basically abstractions of assembly (Fortran, C).\\
Non-Von Neumann architecture: FPGA.\\
\subsubsection{Statements}
A statement is executed by the interpreter to preform an action.
\begin{itemize}
	\item Assignment expressions
	\item Expression statements
	\item Conditional statements
	\item Iterative statements (loops)
\end{itemize}
If, elif and else is part of the same statement, with if being the header of the
statement. In js, if and else if are two separate statements.\\
Flase in pyhton: False, 0 '', None, 0.0, [], \{\}, ().\\
When using ranges, always express as $a\leq i < b$. Where $a=0,b=length$.

\textbf{Strengths:}
\begin{itemize}
	\item Close to machine 
	\item easy to understand 
	\item More efficient than functional
	\item Popular
	\item States
\end{itemize}
\textbf{Disadvantages}: Unexpected behaviour from states, data races.
\subsection{Data structures}
Big performance differences using vectors, deque and lists when inserting at
front or back with different data sizes. Sometimes allocating memory is more
expensive, sometimes moving data is more expensive.\\
Also, access times and sorting times differ.
\begin{description}
	\item[Arrays] Random access, efficient unless resize
	\item[List] Access through iteration, O(1) for insertion
\end{description}
\subsubsection{Tree}
Implement using an array, first element is the value, the next element is an array with children.
Tree traversal using a listener: call entering function, recursive call to all
children, then call exiting function.\\
Tree traversal using a visitor. Recursive call to all children, return depth.
\\Tree recursion arises when executing recursive function that makes multiple
recursive calls. Example: fib(n).

\subsubsection{Dictionary}
In many dynamic object oriented languages object are dictionaries. Found in
obj.\_\_dict\_\_.

\marginpar{\emph{Lecture 4\\2016-02-03}}
\section{Evaluation}
\subsection{Substitution Model}
Applicative order:
\begin{itemize}
	\item Evaluate the arguments and then apply.
\end{itemize}
Normal order:
\begin{itemize}
	\item Fully expand to primitive operations, then reduce
\end{itemize}
\begin{itemize}
	\item Substitution is based on the notion that symbols are names for
		values
\end{itemize}
\subsection{Writing an evaluator}
\subsection{Evaluation Model}
\section{VM and Bytecode}
\begin{verbatim}
LOAD_MEMBER [VARNAME]
/* Is equivalent to */
PUSH [VARNAME]
LOAD ``FUNC_LOAD''
CALL 2 /* VARNAME, OBJECT */
\end{verbatim}

\begin{verbatim}
PUSH 1
PUSH 2
ADD
/* PUSHES 3 to the stack */

LOAD ``a''
PUSH 2
SUB
/* Pushes a-2 to the stack /*

DECL ``a'' # var a;
PUSH 1
STORE ``a'' # a=1

PUSH 2
LOAD ``a''
ADD
STORE ``a''
/* a += 2 */

LOAD ``c''
LOAD ``a''
STORE_MEMBER ``b''
/* a.b = c */


PUSH 1
LOAD ``func''
CALL 1
/* func(1) */


PUSH 2
PUSH 1
LOAD ``func''
CALL 2
/* func(1,2) */

PUSH ``Hello. world!''
LOAD ``console''
LOAD_MEMBER ``log''
CALL 1
/* console.log(``Hello, world!'') */


LOAD ``a'' 			# 1
NOT					# 2
IFJMP 8				# 3
PUSH ``test''		# 4
LOAD ``console''	# 5
LOAD_MEMBER ``log''	# 6
CALL 1				# 7
/* if(a) { console.log('test') }

LOAD ``a''
IFJMP if
/* Else block */
JMP cont
/* If block */
/* Continuation */
/* if(a){console.log(``hello'')} else { console.log(``world'')} */

LOAD ``a''
NOT
IFJMP 6
LOAD ``a''
PUSH 1
SUB
STORE ``a''
JMP 1
/* while(a){a-=1} */

DELC ``a''
PUSH 0
STORE ``a''
LOAD ``a''
PUSH 10
INFERIOR
NOT
IFJMP  # after loop
/* loop body */
LOAD ``a''
PUSH 1
ADD
STORE ``a''
JMP # First LOAD
/* for(var a=0; a<10; a+=1){ console.log(a) } */
\end{verbatim}
\section{Garbage collection and native code}
\marginpar{Lecture 8\\2016-04-05}
\subsection{JIT}
\begin{itemize}
    \item VM recompiles part of program during execution
    \item Adaptive optimization
        \begin{itemize}
            \item Optimize depending on the current context
            \item Profile-guided optimization
        \end{itemize}
\end{itemize}
For dynamic languages it can optimize it for other types. Specialize for int,
string, etc. 
\subsection{From bytecode to native}
\begin{itemize}
    \item Use the registers as local cache
    \item Example:
        \begin{itemize}
            \item Translate to three register machine
        \end{itemize}
\end{itemize}
\begin{verbatim}
PUSH 1
PUSH 1
ADD
PUSH 4
PUSH 5
MUL
DIV
\end{verbatim}
Use register a for stack pointer, b,c for cache.
\begin{verbatim}
SET_B 1
SET_C 2
ADD_B_C
SET_C 4
COPY_MEM_B A
INC_A
SET_B 5
MUL_B_C
DEC_A
COPY_FROM_MEM_C A
DIV_B_C
\end{verbatim}
Generating native code from dynamic typing is harder than static typing.
Some different approaches:
\begin{itemize}
    \item Treat all variables as a pointer to an object
    \item Threat all variables as 32/64 bit integers
        \begin{itemize}
            \item If first bit is 1, then value is object
            \item Else, the value is 31/63 bit integer
        \end{itemize}
    \item Recompile a version of function for all possible types
\end{itemize}
\subsection{V8 case study}
Object representation:
\begin{itemize}
    \item Does not use a dictionary
    \item Creates hidden class definition
\end{itemize}
Create structs, use multiple structs when using dynamic members. Also generate
one native function per parameter type.\\
What about side effects?
\begin{itemize}
    \item A JIT would recompile
\end{itemize}
\subsection{Summary AOT}
\begin{itemize}
    \item Advandages
        \begin{itemize}
            \item Program is ready to use after installation
        \end{itemize}
    \item Disadvandages
        \begin{itemize}
            \item Static optimization slower
        \end{itemize}
\end{itemize}
\subsection{Summary JIT}
\begin{itemize}
    \item Advantages
        \begin{itemize}
            \item Adaptive
            \item More suitable for dynamic programming lang
        \end{itemize}
    \item Disadvantages
        \begin{itemize}
            \item Slow during execution
            \item Debatable if adaptive optimization gives a noticable performance
                improvement
        \end{itemize}
\end{itemize}
\subsection{Summary Native code generation}
\begin{itemize}
    \item Faster
    \item More complexity
    \item Less portable
\end{itemize}
\section{Memory Management}
Two low level operations, malloc() and free(). People forget to free memory,
solution: use a garbage collector.
\subsection{Garbage collector}
\begin{itemize}
    \item Automatically free unused memory
    \item Need allocator and garbage algorithm
\end{itemize}
\subsection{Allocation algorithms}
First approach: One big free space. Fast allocating, slow free due to compacting.
\\Second approach: Free list. Free space is held in a linked list. 
Causes fragmentation. Go through list until you find a big enough space to store new data.
Or find ``best'' fit (smallest available).
\\
When allocation fails:
\begin{itemize}
    \item Do garbage collection and try again
    \item If it doesn't work - increase heap size
    \item Crash with outofmemoryerror
\end{itemize}
\subsection{Garbage collection algorithms}
\subsubsection{Reference counting}
Count the number of pointers pointing to a block.
\begin{itemize}
    \item Pros
        \begin{itemize}
            \item Simple, non-compacting
            \item No need to suspend execution when collecting
        \end{itemize}
    \item Cons
        \begin{itemize}
            \item Extra word in block header
            \item Fragile: bad if counter is forgotten
            \item \textbf{MAJOR} Cannot collect circularly linked data structures.
        \end{itemize}
\end{itemize}
\subsubsection{Mark-Sweep}
Use a marker bit in the memory header. 1 if used, 0 if garbage.
\\When running:
\begin{itemize}
    \item Mark all blocks reachable from root
    \item Sweep all unmarked blocks to the free-list (non-compacting)
\end{itemize}
\begin{itemize}
    \item Pros
        \begin{itemize}
            \item Minimal overhead in memory header
            \item Little heap maintainance
            \item Can collect circular
        \end{itemize}
    \item Cons
        \begin{itemize}
            \item Doesn't deal with heap fragmentation
            \item Suspends the execution
        \end{itemize}
\end{itemize}
\subsubsection{Copying garbage}
Variation of mark-sweep. Copy in use memory to new memory space - compacting.
\textbf{requires 2x the memory}.
\subsubsection{Tri-color marking}
Use three colors
\begin{description}
    \item[White] candidate for deletion
    \item[Black] no connection to white obj and reachable from root
    \item[Gray] reachable from root but with possible reference to white objs
\end{description}
\begin{itemize}
    \item Initialization
        \begin{itemize}
            \item Initial all objects to while and gray
        \end{itemize}
    \item Pick object from gray
        \begin{itemize}
            \item Mark all objects it references to gray
            \item Mark object as black
        \end{itemize}
    \item Stop when gray set is empty
\end{itemize}
This algorithm can be run on-the-fly!
\subsubsection{Generational}
Only but recently created objects in white set.
\end{document}
