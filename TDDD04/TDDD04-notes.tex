
\documentclass[a4paper]{article}

\usepackage[utf8]{inputenc}
\usepackage{float}
\usepackage{hyperref}
\usepackage{listings}
\usepackage{todonotes}
\usepackage{amsmath}
\usepackage{framed}

\author{Pontus Persson}
\title{Lecture Notes\\TDDD04}

\begin{document}
\maketitle
\tableofcontents

\section{What is testing?}

\todo{2016-08-29}{Lecture 1}

\subsection{Types of Oracles}
\begin{description}
  \item[Kiddie Oracles] Just run the program and see what comes out. If it looks
    about righ, it must be right.
  \item[Regression Test Suites] Run the program and compare the output to the
    results of the same tests run against a previous version of the program.
  \item[Validated Data] Run the program and compare the results against a
    standard such as a table, formula or other accepted definition of valid
    output
  \item[Purchased Test Suites] Run the program against a standardized test suite
    that has been previously created and validated. Programs like compilers, Web
    browsers and SQL processors are often tested against such suites.
  \item[Existing Program] Run the program and compare the output to another
    version of the program
\end{description}

\subsection{Testing Levels}
Typically testing is performed at four different levels:
\begin{description}
  \item[Unit Testing] A unit is the smallest piece of software a developer
    creates. In C++ and Java the unit is the class; in C the unit is the
    function.
  \item[Integration Testing]  In integration we assemble units together into
    subsystems. It is possible to units to function perfectly in isolation but
    to fail when integrated (e.g. called with wrong arguments).
  \item[System Testing] Testing at the highest level of integration. Many types:
    functionality, usability, security, internationalization, reliability,
    performance and many more.
  \item[Acceptance Testing] Acceptance testing is defined as that testing, which
    when completed successfully, will result in the customer accepting the
    software and giving us the money.
\end{description}
All these levels are not applicable when testing all systems. In web
applications with a very short development time this is a more common alternate
set of levels:
\begin{itemize}
  \item Cody quality
  \item Functionality
  \item Usability
  \item Performance
  \item Security
\end{itemize}

\subsection{Order of Execution}
There are two styles of test case design regarding order of test execution.
\begin{description}
  \item[Cascading test cases] Tests may build on each other. The next test use
    the state achieved in the program by the previous test.
  \item[Intependent test cases]
\end{description}
\subsection{Limitations of testing}
\label{sec:limitations of testing}

Testing cannot prove correctness.
\begin{itemize}
\item Testing can demonstrate the presence of failures
\item Testing cannot prove the absence of failures
\end{itemize}

\todo{2016-08-31}{Lecture 2}

\section{Black box testing}
Black box testing is a strategy in which testing is based solely on the
requirements and specifications. Unlike white box testing, black box testing
requires no knowledge of the internal paths, structure, or implementation of the
SUT.
\begin{framed}
  \noindent\textbf{Note:} A tester can never be sure of how much of the system has been tested when
  using black box testing.
\end{framed}

Different kinds of unit testing:
\begin{itemize}
  \item Code inspections
  \item Code walk troughs
  \item Black-box
  \item White-box
\end{itemize}
Inputs and how they are constructed must be constructed before-hand.
For design \textit{specification defects}, inspections are both more efficient and more
effective than functional testing. For \textit{code}, functional or structural
testing is ranked more effective or efficient than inspection in most studies.

\subsection{Equivalence class testing}
Equivalence class testing is a technique used to reduce the number of test cases
to a manageable level while still maintaining reasonable test coverage.
\begin{framed}
  \noindent\textbf{Example: Job applicants' age}\\
  We are writing a module for a HR system that decide how to process employment
  applications based on a person's age.
  \begin{table}[H]
    \centering
    \begin{tabular}{r l}
      0-16 & Don't hire\\
      16-18 & Can hire as part-time\\
      18-55 & Can hire as full-time\\
      55-99 & Don't hire
    \end{tabular}
    \caption{Organization new hire's age policy}
    \label{tab:agepolicy}
  \end{table}
  \noindent Example implementation:
  \begin{verbatim}
    if (age >= 0 && age <= 16)
      hire = ``no'';
    if (age >= 16 && age <=18)
      hire = ``part'';
    if (age >= 18 && age <= 55)
      hire = ``full'';
    if (age >= 55 && age <= 99)
      hire = ``NO'';
  \end{verbatim}
  \noindent Clearly we don't have to test all ages $\in [1,16]$. Only one value
  needs to be tested. The same is true for the other rages. They are called
  \textbf{equivalence classes}.
\end{framed}

\begin{table}[H]
  \centering
  \begin{tabular}{|l|l|l|}
    \hline
    EC & Condition & valid/invalid \\\hline
    $EC_0$ & $x \neq y \neq z,x,y,z\in \mathcal{Z}^+$ & valid\\ \hline
    $EC_1$ & two equal, one different and $x,y,z\in \mathcal{Z}^+$& valid
    \\\hline
    $EC_2$ & all equal and positive & valid \\\hline
    $EC_3$ & at least one negative & invalid \\\hline
  \end{tabular}
  \caption{Equivalence classes}
\end{table}

\subsection{Boundary value testing}
Focuses on testing boundaries of equivalence classes. Looking back at the
previous example we note that there is a problem at the boundary edges (16, 18
and 55 are in two equiv. classes).\\
Assume that the age ranges were actually specified as inclusive at the lower
bound and exclusive at the upper bound. The interesting values on or near the
boundaries are:
\begin{itemize}
  \item $\left\{ -1, 0, 1 \right\}$
  \item $\left\{ 15,16,17 \right\}$
  \item $\left\{ 17,18,19 \right\}$
  \item $\left\{ 54,55,56 \right\}$
  \item $\left\{ 97,98,99 \right\}$
\end{itemize}

\subsubsection{Technique}
\begin{enumerate}
  \item Identify equivalence classes
  \item Identify boundaries of each equivalence class
  \item Create test cases for each boundary with one case below, on and above
    the boundary
\end{enumerate}

\subsection{Decision Tables}
Decision Tables are an excellent tool to capture certain kinds of system
requirement s and to document internal system design.

\begin{table}[H]
  \centering
  \begin{tabular}{|l|c|c|c|c|}
    \hline
    & \textbf{Rule 1} & \textbf{Rule 2} & \textbf{Rule 4} & \textbf{Rule 4}
    \\\hline
    \textbf{Conditions} & & & & \\\hline
    Married? & Yes & Yes & No & No \\\hline
    Good Student? & Yes & No & Yes & No \\\hline
    & & & & \\\hline
    \textbf{Actions} & & & & \\\hline
    Discount (\$) & 60 & 25 & 50 & 0 \\\hline
  \end{tabular}
  \caption{Decision table for discount}
  \label{tab:disount}
\end{table}
Condition values can also be non-binary, such as ranges and ``don't care''.\\
After creating the decision table it's time to create test cases, one per rule.
\begin{table}[H]
  \centering
  \begin{tabular}{|l|*{3}{r|}}
    \hline
    \textbf{Test Case ID} & \textbf{Married?} & \textbf{Good Student?} &
    \textbf{Expected Result} \\\hline
    \textbf{TC1} & Yes & Yes & 60 \\\hline
    \textbf{TC2} & Yes & No & 25 \\\hline
    \textbf{TC3} & No & Yes & 50 \\\hline
    \textbf{TD4} & No & No & 0 \\\hline
  \end{tabular}
  \caption{Test cases}
  \label{tab:testcasesdiscount}
\end{table}

\subsection{Pair-wise testing}
\subsubsection{Orthogonal Arrays}
Use orthogonal arrays with variables as columns. One test for each row.
\subsubsection{Allpairs Algorithm}
Algorithm available at satisficing.com. Creates similar but unbalanced table
(with fewer rows) to orthogonal array method.

\section{White box Testing}
General white box testing process:
\begin{enumerate}
  \item Analyze SUT's implementation
  \item Identify paths through SUT
  \item Choose input to take paths. Determine expected output
  \item Run tests
  \item Compare actual output to expected
\end{enumerate}


\end{document}
