\documentclass[course, english]{Notes}

\title{Software Security}
\subject{CS}
\author{Pontus Persson}
\email{ponpe412@student.liu.se}
\speaker{Ulf Karg\'{e}n}
\date{04}{11}{2015}
\dateend{15}{01}{2016}
\place{LiTH}

\begin{document}

\lecture{04}{11}{2015}
\section{Introduction}
\subsection{SQL Slammer}
\begin{itemize}
	\item One UDP package to MSSQL 2003 allowed attacker to run arbitrairy code
	\item 90\% of machines infected in 10min
\end{itemize}
\subsection{Stux}
\begin{itemize}
	\item First cyber warfare weapon
	\item 4 windows zero-days
	\item Targetet centrifuges in Iranian nuclear enrichment facility
	\item Reprogrammed centrifuges to spin them out of control and intercepted comms to tell operators everything was ok
\end{itemize}
\subsection{Software Security today}
\begin{itemize}
	\item 10-15 years ago: most attacks "for fun"
	\item Today: almost exclusively motivated by political or economical gains
	\item Almost exclusively targets client and not server software
	\item PDF-readers, MS Office, etc. Are viable targets
\end{itemize}
\subsection{Common types of defects}
\begin{itemize}
	\item Buffer overflows
	\item Race conditions
	\item Encofing bugs
	\item Double free
	\item \ldots
\end{itemize}


\section{Organization}
\begin{itemize}
	\item 10 lectures
	\item 3 labs
		\begin{itemize}
			\item Pong
			\item Static analysis
			\item Web security
		\end{itemize}
\end{itemize}

\lecture{05}{11}{2015}

\section{Secure SW developent}
Security should not be patched in later. Software considerations must permeate all phases of the software developemnt cycle.
\\ Insert security stuff like pen-test, risk analysis and static analysis in all parts of development.
\subsection{Security requirements}
\begin{itemize}
	\item Reqa are gathered during the initial phase of the software development life cycle
	\item Look at missuse cases
\end{itemize}
\subsubsection{Missuse cases}
\begin{itemize}
	\item Identify threats and countermeasures
	\item Identify how we do \textbf{not} want our program to be used
\end{itemize}
Examples in slides.\\
Goes well together with risk analysis.

\subsection{Risk analysis}
\subsubsection{CORAS}
\begin{enumerate}
	\item Experts and clients decide what system to protext. Draw lo-fi
		image
	\item Define assets, high-level risk analysis and communication flow
	\item Prioritize assets, create consequence scale and likelihood values.
		Create risk evaluation matrix
	\item Create threat diagrams
	\item Estimate risks in the diagram (consequence and likleihood)
	\item Risk evaluation, estimates are confirmed or adjusted
	\item Risk treatment
\end{enumerate}
\subsubsection{Attack trees}
Risk evaluation but from the attackers POV. Very easy, see more in the slides.

\subsection{Software development Lifecycle (SDL)}
The team must complete 16 mandatory secuity activities to comply with the MS SDL
process.

\subsubsection{Pre-SDL}
Education
\begin{itemize}
	\item Threat modeling
	\item Secure cofing
	\item Privacy
\end{itemize}
\subsubsection{Requirements}
\begin{itemize}
	\item Operational envitonment
	\item Quality gates (no compiler warnings when commiting codes, X number
		of manual CR's) to go to next phase
	\item Bug bars (no known critical bugs left in system) overall
	\item What needs to be pentested
	\item Privacy rating
		
\end{itemize}

\subsubsection{Design}
\begin{itemize}
	\item Attaack surface reduction
	\item Threat modeling
\end{itemize}

\subsubsection{Implementation}
\begin{itemize}
	\item Publish list of approved tools
	\item List to be approved by external security advisor
	\item Teams should analyze all functions and APIs that will be used in
		conjunction with the sw development and prohibit those unsafe
	\item All code should be scanned for prohibited functions and APIs
	\item Static analysis of code should be performed
\end{itemize}

\subsubsection{Verification}
\begin{itemize}
	\item Dynamic analysis (monitor problems with memory corruption, user
		privilege issues etc.)
	\item Fuss testing (deliberitely introduce malformed or random data to
		an app during dynamic analysis
	\item Update threat model and attack surface analysis, account for any
		design or implementation changes to the system.
\end{itemize}

\subsubsection{Release}
\begin{itemize}
	\item First point of contact. On call contacts with decision making auth
		avaliable within 24 hours.
	\item Security service plans for code inherited from other groups
	\item Same goes for 3rd party libraries
\end{itemize}

\subsubsection{Final security review}
\begin{itemize}
	\item Independent panel goes through everything
		\begin{itemize}
			\item Pass FSR - g2g
			\item Pass - g2g but fix problems in near future
			\item Not pass - go pass to adress SDL reqs that is not
				fullfilled
		\end{itemize}
	\item If privacy rating P1, independent review
	\item Everything needs to be archived for future projects
\end{itemize}

\subsection{Secure design patterns}
Categorized by abstraction, design or implementation.
\begin{description}
	\item[Architecure level] Focus on high level allocation of responces (Ex
		priv.sep.
	\item[Design level] Internal design och signle component (sec factory)
	\item[Implementation level] Secure logger etc.
\end{description}

\subsubsection{Privilege separation}
Reduce the amount of code that is run as a priviledged user.

\subsubsection{Secure factory}
Seperate the security dependent logic incolved in creating an ovject from the
basic functionalitu of the created object

\subsubsection{Secure chain of responsibility}
Decouple the logic that determines pivileges from the portion of the program
that is requesting the privileges.

\subsubsection{Secure logger}
Prevent attacker from gathering potentially useful information about the system
from system logs or editing logs.

\subsubsection{Clear sensitive information}
Is it possible that sensitive information has been stored in reusable resources
after a user session or applicatin has run.

\lecture{06}{11}{2015}

\section{Vulnerabilities in C/C++}
Problem with programs that compile to machine code. 
\subsection{Assembly language primer}
\begin{itemize}
	\item All processes see 4GB of private continuóus virtual memory
	\item The stack is located high and grows down
	\item Main exe (text) nad its Data and BSS segment is located in low
		memory
	\item The heap i located above Text, Data and BSS. Grows upwards.
		\begin{itemize}
			\item Used for dynamically allocated memory (malloc,
				new)
		\end{itemize}
	\item x86 is a little-endian architecture, First byte of e.g. a 4-byte
		word is the least significant byte.
\end{itemize}

\lecture{12}{11}{2015}
\subsubsection{More vulnerabilities}
\begin{itemize}
	\item Integer overflow Wrap-arounds cause length check to pass
	\item Type conversions, usigned to signed, int to short, etc.
\end{itemize}
\subsection{Non memory corruption vulnerabilities}

\begin{itemize}
	\item Race conditions (time between access check and write)
	\item Out of bounds reads (heartbleed)
\end{itemize}

\lecture{13}{11}{2015}

\section{Software Engineering Reviews}
Inspections are the best way to find defects in code and other documents. At
least looking at the number of anomalies found.\\
Used for:
\begin{itemize}
	\item Find defects
	\item Training
	\item Communication
	\item Hostage taking
\end{itemize}

\subsection{Fagan review}
\subsubsection{Process}
\begin{itemize}
	\item Initial
		\begin{itemize}
			\item Check criteria
			\item Plan
			\item Overview
		\end{itemize}
	\item Individual
		\begin{itemize}
			\item Preparation, or
			\item Detection
		\end{itemize}
	\item Group
		\begin{itemize}
			\item Detection, or
			\item Collection
			\item Inspection record
			\item Data collection
		\end{itemize}
	\item Exit
		\begin{itemize}
			\item Change
			\item Follow-up
			\item Document \& data handling
		\end{itemize}
\end{itemize}

\subsubsection{Inspection record}
\begin{itemize}
	\item Identification
	\item Location
	\item Description
	\item Decision for entire document
		\begin{itemize}
			\item Pass w/ changes
			\item Reinspect
		\end{itemize}
\end{itemize}
\subsubsection{Data collection}
\begin{itemize}
	\item Number of defects
	\item Classes of defects
	\item Severity
	\item Other statistics
\end{itemize}
\lecture{18}{11}{2015}

\section{Static Analysis}
Static program analysis analyses computer programs statically, without executing
them.
\begin{itemize}
	\item No need to run, before deployment
	\item No need to restrict to single input for testing
	\item Usefull in compiler optimization, finding security analysis
	\item Ofter on source code, sometimes on object code
	\item Usually highly automated (w/ possibility of some user
		interraction)
\end{itemize}
A sounds analysis algorithm is sound in the case where each time it reports the
program is safe wet. some errors, then the original program is indeed safe wrt.
those errors. An algorithm is complete if it each time its given a program that
is safe with some errors and it reports it save with those errors.
\\
We try to approximate all possible configurations, this gives us an over- or
under approximation.
\begin{itemize}
	\item A sounds analysis cannot give false negatives
	\item A complete analysis cannot give false positives
\end{itemize}
\subsection{Syntactic Analysis}
\subsubsection{Splint}
Splint is neither sound nor complete. Kind of a fancy version of grep. Looks for
``dangerous'' patterns.
\subsection{Abstract interpretation}
Suppose you have a program analysis that captures the progra behaviour but that
is inefficient or uncomputable. You want an analysis that is efficient but that
also can over-approcimate all behaviours of the program.
\\
Consider a language with multiplication, addition and subtraction of integers.
If you are only interested in the sign of variable, you can associate
$\set{+,0,-}$ for each configuration instead of $\mathcal{Z}$. \\\
For an integer variable, the set of concrete values at a location is in
$\mathcal{P}(\mathcal{Z})$. $S_1 \sqsubseteq S_2$ means that $S_1$ is more
precise than $S_2$. Example; $\mathcal{P}(\set{0,+}) \sqsubseteq
\mathcal{P}(\set{+,0,-})$
\subsubsection{Lattice}
A pair $(Q, \leq)$ is a lattice if each pair p,q in Q has
\begin{itemize}
	\item a greatest lower bound $p \sqcap q$ with $\leq$ (meet) and
	\item a least upper bound $p \sqcup q$ with $\leq$ (join)
\end{itemize}
\subsubsection{Galois connections}
$(\alpha,\gamma)$\\
Concretization function $\gamma$, abstraction function $\alpha$. Eg, S is all
prime numbers. $\alpha(S)
= \set{+}=A$ and $\gamma(A)=\set{i|i>0}$. \\
Also, $S\sqsubseteq_c \gamma \circ
\alpha(S)$ and $\alpha \circ \gamma(A) \sqsubseteq_a A$.
\\
\lecture{19}{11}{2015}
\subsection{Symbolic execution}
Symbolic execution testing is complete, but not sound.
\begin{description}
	\item[SMT]Satisfiability Modulo Theory
\end{description}
Similar to SAT solvers, SMT solvers work on ranges and not just boolean values.
\begin{itemize}
	\item Most common form of software validation
	\item Explores only one possible execution at a time
	\item For each new value, run a new test
	\item On 32-bit machine an integer equals test requires $2^{32}$
		different tests
	\item The ida in symbolic testing is to associate \textbf{symbolic
		values} to the variables
	\item Along the path, maintain a Path Constrain (PC) and symbolic state
		($\sigma$)
	\item PC collects constrains on variable's values along a path
	\item $\sigma$ associates variables to symbolic expressions
	\item We get concrete values if PC i satisfiable
	\item Negate a condition in the path constraint to get another path
\end{itemize}

\subsection{Hoares and deductive reasoning}
Total correctness requires that:
\begin{itemize}
	\item If the pre-condition holds
	\item then the implementation terminates
	\item after termination, the post-condition holds
\end{itemize}
A Hoare triple $\set{P}$ stmt $\set{R}$ consists in:
\begin{itemize}
	\item a predicate pre-condition $P$
	\item an instruction stmt
	\item A predictate post-condition $R$
\end{itemize}
\subsubsection{Weakest precondition}
\begin{itemize}
	\item if $\set{P} stmt \set{R}$ and $P' \Rightarrow P$ s.t. $\set{P'}
		stmt \set{R}$, then P is the weakest precondition of R with stmt
		written wp(stmt, R)
\end{itemize}
$wp(x=x+1,x>1,x\geq 1)=(x\geq 0$.
\begin{itemize}
	\item $wp(x=E,R)=R[x/E]$, i.e. replace each occurence of x in R by E.
\end{itemize}
For instance
\begin{align*}
	&wp(x=3,x==5)=(x == 5)[x/3]=(3 == 5) == false \\
	&wp(x=3,x>0)=(x>0)[x/3]=(3 >0) == true \\
	&wp(stmt;stmt',R) = wp(stmt, wp(stmt', R))
	\label{Weakest precondition}
\end{align*}
In order to establish $\set{P} (while(B)do \set{ stmt }) \set{R}$, we need to fins
an invariant $Inv$:
\begin{itemize}
	\item $P \Rightarrow Inv$
	\item $\set{ Inv && B }stmt \set{ Inv }$
	\item $(Inv&&!B) \Rightarrow R$
\end{itemize}

\end{document}
