\documentclass[a4paper]{article}
\usepackage[utf8]{inputenc}
\usepackage{float}
\usepackage{hyperref}
\usepackage{amsthm}
\usepackage{todonotes}
\usepackage{amsmath}
\usepackage{framed}

\author{Pontus Persson}
\title{Lecture Notes\\TDDC17}
\date{HT-16}

\newtheorem{definition}{Definition}[section]

\begin{document}
\maketitle
\tableofcontents
\clearpage

\section{Introduction}

\todo{2016-08-29}{Lecture 1}

\subsection{What is AI?}
\textbf{Agent}-based view. An agent interacts with its environment through 
sensors and actuators. It implements an agent function which maps any percept
sequence to an action.
\subsection{Two approaches to AI}
The goal of AI can be either to achieve an intelligence that matches human
thinking very closely or an intelligence that thinks or behaves rational.
\begin{table}[H]
  \centering
  \begin{tabular}{|l|l|l|}
    \hline
    & Human-Centered & Rationality-Centered \\ \hline
    Thought process reasoning & & \\ \hline
    Behavior & & \\ \hline
  \end{tabular}
  \caption{AI approaches matrix}
  \label{tab:aiapproachesmatrix}
\end{table}

\section{Intelligent Agent Paradigm}

A rational agent is one that does the right thing relative to an external
performance metric. The human body can be thought of as an intelligent agent.

\subsection{Evolutionary AI}

Introduce a progression of agents (AI systems) each more complex than its
predecessor. Incrementally introduce new techniques to exploit information from
the environment directly not sensed.

\subsection{Rationality}
Rationality is dependant on:
\begin{itemize}
  \item The agent's percept sequence; everything the agent has received so far
  \item The embedding environment; what the agent knows about its environment
  \item The agent's capabilities; the actions it has available
  \item The external performance measure used to evaluate the agent's
    performance
\end{itemize}

\begin{definition}[Ideal Rational Agent]
  For each possible percept sequence, an ideal rational agent should do whatever
  action is expected to maximize its performance measure, on the basis of the
  evidence provided by the percept sequence and whatever built-in knowledge the
  agent has.
\end{definition}

\subsection{Task Environments}
\begin{description}
  \item[Fully or partially observable] An agent's sensory apparatus provides the
    \textit{complete} state of the environment
  \item[Deterministic or stochastic] The next state of the environment only
    depends on its previous state and the agent's actions
  \item[Static or dynamic] The environment is unchanged while the agent is
    deliberating
  \item[Descrete or continuous] There are a limited number of distinct percepts
    and actions. States can be descrete or continuous
  \item[Episodic or sequential] The agent's experience is divided into episodes
    such as ``perceiving and acting''. The quality of the action is only
    dependent on the current episode
  \item[Single or multiple agent(s)] An environment can contain multiple agents
    working cooperatively or competitively
\end{description}

\subsection{Types of intelligent agents}

\subsubsection{Simple Reflex Agent}
Also called \textit{Stimulus-Response Agent}.
\begin{itemize}
  \item Reacts immediately to stimuli in their environment
  \item Has no internal state
  \item Uses current state of the environment derived from sensory stimuli
\end{itemize}

\todo{2016-08-30}{Lecture 2}
\vspace{5mm}
\begin{framed}
\noindent \textbf{Example: navigate}\\
Specify a function of the sensory inputs that selects actions appropriate for
task ahievement.
\begin{align}
  f: \left\{ s_1,s_2,s_3,s_4,s_5,s_6,s_7,s_7 \right\} \rightarrow \left\{ north,
    east, south, west
  \right\}
\end{align}

Perception processing produces a set of features ($x_1,..,x_n$) from a set of
sensor values ($s_1,..,s_n$).
\begin{align}
  x_1=s_2+s_3\\
  x_2=s_4+s_5\\
  x_3=s_6+s_7\\
  x_4=s_8+s_1\\
\end{align}
Action function phase:
\begin{align}
  x_1 \land \lnot x_2 &\mbox{ east} \\
  x_2 \land \lnot x_3 &\mbox{ south} \\
  x_3 \land \lnot x_4 &\mbox{ west} \\
  x_4 \land \lnot x_1 &\mbox{ north} \\
  \mbox{else} &\mbox{ north}
\end{align}
\end{framed}
\subsubsection{Model-based Reflex Agent}
\begin{itemize}
  \item Limited internal state
  \item Environmental state at $t+1$ is a function of:
    \begin{itemize}
      \item the sensory input at $t+1$
      \item the action taken at time $t$
      \item the previous environmental state at $t$
    \end{itemize}
\end{itemize}

If all important aspects of the environment relevant to a task can be sensed at
the time the agent needs to know them and no need for internal memory.\\
In general, sensory capabilities are limited. Compensate by using a stored model
of the environment.

\subsubsection{Goal-Based Agents}
Planning and reasoning agents. \\
Main part of course:
\begin{itemize}
  \item Search
  \item Knowledge representation and reasoning
  \item Planning
\end{itemize}
Agents with purpose!
\begin{itemize}
  \item Rich internal state
  \item Can \textbf{anticipate} the affects of their action
  \item Take those actions expected to lead toward achievement of goals
  \item Capable of reasoning and deducting properties of the world
\end{itemize}

\subsubsection{Utility-Based Agents}
Similar to goal based agents but with preferences about future states in
environment.

\subsubsection{Learning-Based Agents}
\begin{itemize}
  \item Has the ability to modify behaviour for the better based on experience
  \item It can learn new behaviours via exploration of new experiences
\end{itemize}

\subsection{Deliberation vs. Reaction}
More information $\Rightarrow$ slower response, but better predictions and
accurate\\
Less information $\Rightarrow$ fast response, no predictions.\\
\textbf{A good AI is both deliberate and reactive}.

\section{Physical symbol system}
\begin{definition}[Physical symbol system]
  A physical symbol system consists of:
  \begin{itemize}
    \item A set of entities called symbols which are physical patterns that can
      occur as components of another type of entity called an expression (or
      symbol structure)
    \item At any instant of time the system will contain a collection of symbol
      structures.
    \item The system also contains a collection of processes that operate on
      expressions to produce other expressions: processes of creation,
      modification reproduction and destruction.
  \end{itemize}
  
\end{definition}
A physical symbol system has the necessary and sufficient means for general
intelligent action.
\\
\textbf{WILL BE ON EXAM!, read article!}
\\
\\
\todo{2016-08-31}{Lecture 3}

\section{Search}

Proof of optimality of A* using three-search will be on the exam.

\begin{proof}{A*}
  Insert proof from slides here
\end{proof} 

\todo{2016-09-06}{Lecture 4}

\todo{2016-09-13}{Lecture 6}

\section{Propositional logic}
\textbf{Atoms:} Two distinguished atoms T and F\\
Semantics is about associating elements of a logical language with elements of a
domain of discourse. In the case of propositional logic the domain of discourse
is propositions about the world. An interpretation associates a proposition with
each atom and a value (True or False).\\
An interpretation satisfies a well formed formula if the wff is assigned the
value True under the interpretation.\\
A wff is said to be inconsistent or unsatisfiable if there are no
interpretations that satisfy it.\\
A wff is said to be valid if it has value True under all interpretations of its
constituent atoms. Example $\not (P \land P)$.\\

\begin{align}
  \lnot (P \land Q) &\Leftrightarrow \lnot P \lor \lnot Q
\end{align}
If a wff $\omega$ has a value True under all those interpretations for which
each of the wffs in a set $\Delta$ has value True, then we say that $\Delta$
logically entails $\omega$ $\Delta \models \omega$.\\
$\omega_n$ can be proved from a set $\Delta$: $\Delta \vdash \omega_n$
\section{Soundness and Completeness}
\begin{framed}
  If, for any set of wffs $\Delta$, and wff $\omega$, $\Delta \vdash \omega$
  implies $\Delta \models \omega$, we say that the set of inference rules is
  sound.
\end{framed}

\begin{framed}
  If for any set of wffs $\Delta$ and wff, $\omega$ it is the case that whenever
  $\Delta \models \omega$, there exists a proof of $\omega$ from $\Delta$ using
  the set of inference rules, we say that it is complete
\end{framed}


\end{document}
