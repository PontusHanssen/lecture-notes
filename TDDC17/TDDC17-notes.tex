%        File: TDDC78-lecture-notes.tex
%     Created: Mon Apr 04 08:00  2016 C
% Last Change: Mon Apr 04 08:00  2016 C
%
\documentclass[a4paper]{article}
\usepackage[utf8]{inputenc}
\usepackage{float}
\usepackage{hyperref}
\usepackage{amsthm}
\usepackage{todonotes}

\author{Pontus Persson}
\title{Lecture Notes\\TDDC17}
\date{HT-16}

\newtheorem{definition}{Definition}[section]

\begin{document}
\maketitle
\tableofcontents

\section{Introduction}

\todo{2016-08-29}{Lecture 1}

\subsection{What is AI?}
\textbf{Agent}-based view. An agent interacts with its environment through 
sensors and actuators. It implements an agent function which maps any percept
sequence to an action.
\subsection{Two approaches to AI}
The goal of AI can be either to achieve an intelligence that matches human
thinking very closely or an intelligence that thinks or behaves rational.
\begin{table}[H]
  \centering
  \begin{tabular}{|l|l|l|}
    \hline
    & Human-Centered & Rationality-Centered \\ \hline
    Thought process reasoning & & \\ \hline
    Behavior & & \\ \hline
  \end{tabular}
  \caption{AI approaches matrix}
  \label{tab:aiapproachesmatrix}
\end{table}

\subsection{Rationality}
Rationality is dependant on:
\begin{itemize}
  \item The agent's percept sequence; everything the agent has received so far
  \item The embedding environment; what the agent knows about its environment
  \item The agent's capabilities; the actions it has available
  \item The external performance measure used to evaluate the agent's
    performance
\end{itemize}

\begin{definition}[Ideal Rational Agent]
  For each possible percept sequence, an ideal rational agent should do whatever
  action is expected to maximize its performance measure, on the basis of the
  evidence provided by the percept sequence and whatever built-in knowledge the
  agent has.
\end{definition}

\section{Intelligent Agent Paradigm}

A rational agent is one that does the right thing relative to an external
performance metric. The human body can be thought of as an intelligent agent.

\subsection{Evolutionary AI}

Introduce a progression of agents (AI systems) each more complex than its
predecessor. Incrementally introduce new techniques to exploit information from
the environment directly not sensed.

\subsection{Task Environments}
\begin{description}
  \item[Fully or partially observable] An agent's sensory apparatus provides the
    \textit{complete} state of the environment
  \item[Deterministic or stochastic] The next state of the environment only
    depends on its previous state and the agent's actions
  \item[Static or dynamic] The environment is unchanged while the agent is
    deliberating
  \item[Descrete or continuous] There are a limited number of distinct percepts
    and actions. States can be descrete or continuous
  \item[Episodic or sequential] The agent's experience is divided into episodes
    such as ``perceiving'' and ``acting''. The quality of the action is only
    dependent on the current episode
  \item[Single or multiple agent(s)] An environment can contain multiple agents
    working cooperatively or competitively
\end{description}

\subsection{Types of intelligent agents}

\subsubsection{Simple Reflex Agent}
\begin{itemize}
  \item Reacts immediately to stimuli in their environment
  \item Has no internal state
  \item Uses current state of the environment derived from sensory stimuli
\end{itemize}

\end{document}
