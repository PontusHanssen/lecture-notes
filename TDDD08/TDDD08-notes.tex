
\documentclass[a4paper]{article}

\usepackage[utf8]{inputenc}
\usepackage{float}
\usepackage{hyperref}
\usepackage{listings}
\usepackage{amsmath}

\author{Pontus Persson}
\title{Lecture Notes\\TDDD08}

\begin{document}
\maketitle
\tableofcontents

\section{Introduction}
\label{sec:introduction}
\textbf{At exam:} One a4 paper of own notes.\\
Core pillars of cource:
\marginpar[2016-08-29]{Lecture 1}
\begin{itemize}
\item Declarative Semantics
\item Operational Semantics
\item Programming Semantics
\item Grammars (definite closed grammars)
\item Negation in logic programming
\item Constrains logic programming
\end{itemize}

\subsection{Cource objectives}
\begin{itemize}
\item Logic as a programming language
\item Theoretical foundation of LP
\item prolog
\item Program/think \textbf{declaratively}
\end{itemize}

\subsection{Declarative vs. Imperative languages}
Imperative languages descibe actions of a machine. Thinking in terms of a Von
Neuman machine. Basic concept - variable, abstraction of a RAM cell.

\begin{align}
 x=x+1\\
\mbox{In mathematics: }x_{i+1} = x_i+1
\end{align}
\\
A declarativ program describes what should be computed, not necessary how.
\\It describes the problem/solution - closer to human thinking. Variables are
like in mathematics.
\subsection{Logical Programming}
\begin{description}
\item[Program] a set of axioms
\item[Result] its logical consequences
\item[Computation] proof construction
\end{description}
Main programming language - \textit{prolog}.\\

Find grandchild of $X$ using logical proramming notation.
\begin{align}
  child(X,Y)  \mbox{ //X is child of Y}\\
  grandchild(X,Z) \leftarrow child(X,Y) \land child(Y,Z)
\end{align}

\subsection{Two levels of reading a program}
\begin{description}
\item[declarative] a set of axioms
\item[operational] a description of computations
\end{description}
\begin{align}
 ALGORITHM = LOGIC + CONTROL 
\end{align}
Control information
\begin{itemize}
\item The order of axioms and within axioms
\item some extra constructs
\end{itemize}
The two levels can be considered seperately. Program correctness is a property
of the declarative level. But operational level affects performance.
\\
Programs consists of rules and facts:
\begin{align}
  p(...)\leftarrow p_1(...),...,p_n(...)
\end{align}
In prolog $\leftarrow$ is written as \texttt{:-}.\\

\section{Main concepts of logic}
\begin{description}
\item[Constants] numbers, lower case strings
\item[Function symbols] cons/2, +/2, father/1
\item[Variables] Starts with upper case
\item[Logical connectives] $\land, \lor, \rightarrow, \lnot, \leftrightarrow$
\item[Quantifiers] $\forall, \in$
\item[Auxillary symbols] $.,(,),...$
\end{description}

Ground term (formula) - containing no variables.\\
\end{document}